\documentclass[12pt]{article}

\title{A review on database sequence search}
\author{Fabian Bull}
\date{\today}

\usepackage[style=numeric,backend=biber]{biblatex}

\addbibresource{bib/bib.bib}

\begin{document}
\maketitle

\section{Introduction}
\label{sec:introduction}

Searching for similarities between a pattern and a potentially large text is a well studied field in bioinformatics and computer science in general.
This review tries to give an overview of the last two decades of the field, beginning at the seminal work of \cite{Altschul1990} to the last break throughs of \cite{Buchfink2015}.
While the algorithms and approaches presented here, are not constrained to biological applicationr, I stick to the biological use case to provide a complete picture of the problem domain.

\section{The approach}
\label{sec:approach}

When searching for a pattern \emph{p} in a larger text \emph{T}, we often release the constraint that a query has to exactly match a text in the database.
This problem is often called \emph{approximate string matching}. 
Instead of searching for exact matches, a limited number of differences \emph{k} between the pattern and its occurrences in the text are allowed.
The oldest solution \cite{Sellers1980} to the problem relies on \emph{dynamic programming} (often just called DP). 
While this approach allows for more complex distances, its $\mathcal{O}(n^2)$ performance makes it prohibitiv for larger data sets.

In todays algorithms, the goal is, to apply dynamic programming only when we already know that there is a match.
This approach is first described in the BLAST program \cite{Altschul1990}.
It can be used as a blue brint for all algorithms thereafter.

\subsection{Indexing}
\label{subsec:indexing}

The process of indexing builds the data structres that speed-up the seeding done afterwards.

\begin{itemize}
    \item HashMaps
    \item SuffixArrays \cite{Burkhardt1999}
    \item FM-Index
    \item Bidirectional FM-Index \cite{Buchfink2015}
\end{itemize}

\subsection{Seeding}
\label{subsec:seeding}



\begin{itemize}
    \item Motivate why seeding is needed.
    \item What is a seed?
    \item Explain Seed and extend -> Verification
\end{itemize}

\subsubsection{Contiguous Seeds}
\label{subsubsec:cont-seeds}

\subsubsection{Spaced Seeds}
\label{subsubsec:spaced-seeds}

\begin{itemize}
    \item Are superios \cite{Zhang2007}
\end{itemize}

\subsubsection{Goodness of a Seed}
\label{subsubsec:goodness-seed}

\subsubsection{Extensions}
\label{subsubsec:seeds-extensions}

\subsubsection{Open Questions}
\label{subsubsec{seeds-questions}

\begin{itemize}
    \item vector seeds
\end{itemize}


\subsection{Verification}
\label{subsec:verification}

Once a seed matches a the text, the verification is done using DP. 
This is because of the complex distances needed for a realistic comparison of the pattern and its match.

\section{Conclusion}
\label{sec:conclusion}

\section{References}
\label{sec:references}

\printbibliography


\end{document}
